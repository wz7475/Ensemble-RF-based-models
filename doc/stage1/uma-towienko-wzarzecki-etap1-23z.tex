% !TeX spellcheck = pl_PL
\documentclass[10pt,a4paper]{article}

\usepackage{geometry}
\geometry{
	a4paper,
	total={170mm,257mm},
	left=20mm,
	top=20mm,
}

\usepackage{mathtools}
\usepackage{polski}
\usepackage{amsfonts}
\usepackage{amsmath}
\usepackage[T1]{fontenc}
\usepackage{listings}
\usepackage{booktabs}
\usepackage{csvsimple}
\usepackage{geometry}
\usepackage{siunitx}
\usepackage{caption}
\usepackage{booktabs}
\usepackage{longtable}
\usepackage{float}
\usepackage{algpseudocode}
\usepackage{algorithm}
\usepackage{hyperref}



\captionsetup[table]{skip=10pt}

\sisetup{round-mode=places, round-precision=3, round-integer-to-decimal, scientific-notation = true, table-format = 1.3e2, table-number-alignment=center, group-separator={,}}


\title{Uczenie Maszynowe - projekt}
\author{Tomasz Owienko \and Wojciech Zarzecki}
\date{18.11.2023}


\begin{document}
	
\maketitle	


\section{Cel projektu}

Celem projektu jest implementacja zmodyfikowanej wersji algorytmu generowania lasu losowego, w której do generowania kolejnych drzew losowane są częściej elementy ze zbioru uczącego, na których dotychczasowy model się mylił.

Istotą metody lasu losowego w problemach regresji i klasyfikacji jest redukcja wariancji i nadmiernego dopasowania osiąganego przez pojedyncze drzewa decyzyjne. W klasycznych algorytmach generowania lasu losowego każde z $B$ drzew generowane jest na podstawie $\sqrt{B}$ przykładów ze zbioru trenującego wylosowanych ze zwracaniem, zazwyczaj ograniczonych (w problemach klasyfikacji) do $\sqrt{|D|}$ atrybutów wylosowanych bez zwracania, gdzie $D$ jest zbiorem atrybutów. Proces ten odbywa się w jednej iteracji - algorytm kończy pracę po wygenerowaniu $B$ drzew. Taka koncepcja to tzw. \textit{bagging}, który ma na celu ograniczenie wariancji modelu po przez agreację wielu prostszych modeli - w tym przypadku drzew decyzyjnych.

Realizowany projekt zakłada zbadanie możliwości zastosowania metod \textit{boostingu} oraz \textit{stackingu}. Wdrożenie metod boostingowych polega na uwzględnianiu błędów poprzedniego drzwewa decyzyjnego, przy budowanie kolejnego. W takim podejściu nowe drzewa są nieustannie dołączane są do istniejącego już lasu. Natomiast stacking zakłada utworzenie nowego zbioru danych na bazie wyników wytrenowanego modelu oraz wytrenowanie na nowym zbiorze \textit{metamodelu}. Co istotne, w podejściu stackingowym las wygenerowany w iteracji $n+1$ w pełni zastępuje las z iteracji $n$.

\section{Opis algorytmu}

Algorytm korzystający z boostingu - kolejne drzewa dołączane są cały czas do tego samego zbioru:

\begin{algorithm}[h]
	\caption{Boosted Random Forest}\label{alg:caprf_train}
	\hspace*{\algorithmicindent} \textbf{Input}: $U\neq \emptyset$: zbiór przykładów trenujących, $C$: klasy przykładów, $D$: zbiór atrybutów wejściowych
\begin{algorithmic}
	\State 	// przy założeniu, że każda iteracja to nowy las
	\State $F \gets \emptyset$
	\State $\hat C \gets \emptyset$
	\For{$i \gets 1$ to $N$} 
		\State $F_i \gets \emptyset$
		\For{$b \gets 1$ to $B$} 
			\State $U_b \gets $B elementów wylosowanych z $U$ z preferencją dla $\{u_j \in U: \hat C(u_j) \neq C(u_j)\}$
			\State $D_b \gets \sqrt{|D|} $ atrybutów wylosowanych z $D$ bez zwracania
			\State $f_b \gets $ drzewo decyzyjne wygenerowane na podstawie $C$, $U_b$ i $D_b$
			\State $F_i \gets F_i \cup \{f_b\}$
		\EndFor
		\State $F \gets $ las losowy $F$ ulepszony za pomocą $F_i$
		\For{$u \in U$}
			\State $\hat C \gets \hat C\ \cup$ PredictRandomForest($u$, $F$)
		\EndFor
		
	\EndFor
	\State \Return $F$
\end{algorithmic}
\end{algorithm}

Algorytm korzystający ze stackingu:


\begin{algorithm}[H]
    \caption{Stacked Random Forest with Iterative Retraining}\label{alg:iterative_stacked_rf}
    \hspace*{\algorithmicindent} \textbf{Input}: $U\neq \emptyset$: zbiór przykładów trenujących, $C$: klasy przykładów, $D$: zbiór atrybutów wejściowych
    \begin{algorithmic}
        \State $OriginalData \gets U$
        \For{$m \gets 1$ to $M$}
            \State // Trenowanie N niezależnych lasów losowych
            \State $Forests \gets \emptyset$
            \For{$i \gets 1$ to $N$}
                \State $F_i \gets $ RandomForest($OriginalData, C, D$)
                \State $Forests \gets Forests \cup \{F_i\}$
            \EndFor
            \State // Tworzenie nowego zestawu danych na podstawie wyników lasów
            \State $NewData \gets \emptyset$
            \For{$u \in OriginalData$}
                \State $Results \gets []$
                \For{$F_i \in Forests$}
                    \State $Results \gets Results\ \cup$ PredictRandomForest($u, F_i$)
                \EndFor
                \State $NewData \gets NewData\ \cup \{(Results, C(u))\}$
            \EndFor
            \State $OriginalData \gets NewData$
        \EndFor
        \State \Return $Forests$
    \end{algorithmic}
\end{algorithm}

Predykcja za pomocą wygenerowanych lasów losowych - bez różnic względem podejścia klasycznego:

\begin{algorithm}
\caption{PredictRandomForest}\label{alg:rf_predict}
\hspace*{\algorithmicindent} \textbf{Input}: $x$: wektor wejściowy, $F$: nauczony las losowy
\begin{algorithmic}[H]
	\State $\hat C \gets \emptyset$
	\For{$f_b \in F$} 
		\State $\hat C \gets \hat C \cup {f_b(x)}$
	\EndFor
	\State \Return najczęstsza klasa z $\hat C$
\end{algorithmic}
\end{algorithm}

\section{Planowane eksperymenty}\
Planujemy porównać efektywność następujących podejść:

\begin{itemize}
    \item klasyczny Random Forest
    \item Boosted Random Forest
    \item Stacked Random Forest
    \item połącznie Stacked RF i Boosted RF
\end{itemize}


\section{Wykorzystywane zbiory danych}
\begin{itemize}
    \item \href{https://archive.ics.uci.edu/ml/datasets/Mushroom}{UCI Mushrooms Dataset}
    \item \href{https://archive.ics.uci.edu/ml/datasets/Breast+Cancer}{UCI Breast Cancer Dataset}
    \item \href{https://www.kaggle.com/datasets/vinicius150987/titanic3}{Titanic Dataset on Kaggle}
\end{itemize}


\end{document}